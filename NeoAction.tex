\documentclass[10pt,a5paper]{article}

%idioma español y acentos
\usepackage[spanish]{babel}
\usepackage[utf8]{inputenc}

\usepackage{geometry}
\geometry{a5paper, margin=0.75in}

\usepackage{fancyhdr}
 
\pagestyle{fancy}
\fancyhf{}
\rhead{\textbf{NEOILLUMINATI ACTION}}
\lhead{\leftmark}
\rfoot{Página \thepage}

\title{\textbf{Juego de beber definitivo\\NEOILLUMINATI ACTION}}
\author{Adrián Alberdi Ainciburu\\
		Jose Daniel Jimenez Delgado}
\date{}
\begin{document}

\begin{titlepage}
\maketitle
\end{titlepage}

\section{Descripción del juego}
Este es un juego de beber solo apto para alcohólicos extremos. Es un juego en el que todos ganan ya que no tiene ningún objetivo más allá de beber por beber. No hay forma de acabar el juego, así que las partidas pueden llevar días o una vida entera.
\section{Piezas}
El juego consta de las siguientes partes:
\begin{itemize}
	\item 1x Tablero
	\item x Cartas verdes
	\item x Cartas naranjas
	\item x Cartas rojas
	\item 2x Dados
	\item 1x Ruleta
	\item 12x Vasos de chupito
	\item 1x Bombo de Bingo
	\item 10x Cartones de Bingo
	\item 1x Baraja francesa
\end{itemize}
\section{Modo de Juego}
Se trata de un juego de mesa de estilo tradicional mezclado con la ruleta de chupitos, el bingo alcohólico y el juego del conductor del autobus. \\
Al comienzo de la partida cada jugador deberá coger una ficha reversible y un cartón de bingo. Esto a parte, por supuesto, de un cubata o una cervecita para refrescarse la garaganta durante la larga partida. Cada jugador guardará su cartón de bingo con cuidado y colocará su ficha en la casilla de salida.\\
El juego básico es tirar un dado avanzar el número de casillas dictadas por el dado. Según donde caiga el jugador deberá hacer lo que dicte la casilla, se puede ver en la sección \ref{sec:casillas}.\\
Cada vez que un jugador pase por la casilla de salida podrá decidir si se duplica el número de chupitos. En caso de decidir duplicar el contador se deberá avanzar.\\
En la pizarra hay que apuntar cada bebida/chupito por motivos de control y para saber quien bebe menos.

\section{Tablero}\label{sec:casillas}
El tablero se dispondrá sobre una superficie plana. Sobre las marcas se situará la ruleta, los chupitos, el bingo, los cartones del bingo y las cartas de colores. El contador se colocará en el primer círculo.
La descripción de las casillas de las que consta el tablero se puede ver a continuación.

\subsection{Ruleta}
Se tira la bolita y se bebe según en que tercio entre. En caso de caer en el 00 se han de beber todos!!

\subsection{Bingo}
Se saca la bola del bombo. Toda persona que tenga el número bebe 1 trago excepto en casos especiales descritos a continuación:
\begin{enumerate}
	\item \textbf{El jugador hace linea:} El jugador tiene derecho a mandar 3 chupitos. Los chupitos se han de repartir entre diferentes jugadores.
	\item \textbf{El jugador hace bingo:} El jugador tiene derecho a mandar 5 chupitos. Los chupitos se han de repartir entre diferentes jugadores. El jugador Recibirá un nuevo cartón antes de continuar.
\end{enumerate}

\subsection{Casilla escalera/tobogán}
El jugador seguirá la escalera/tobogán hasta la casilla destino procediendo ya en el destino a seguir las instrucciones de la casilla.

\subsection{Estación}
El jugador procederá a jugar una ronda de autobús. Este juego consiste en adivinar 4 cualidades de 4 cartas de la baraja francesa. Las cualidades serán las siguientes en este orden:
\begin{enumerate}
	\item \textbf{Color:} Rojo o negro
	\item \textbf{Por encima o por debajo:} Si la carta que saldrá es mayor, menor o igual a la ya sacada, siendo el As la carta de mayor valor de la baraja
	\item \textbf{Dentro o fuera:} Si la carta se encontrará entre las dos cartas anteriores un valor externo o el mismo que las anteriores
	\item \textbf{Igual o diferente:} Si el palo de la carta será igual o diferente a alguno de los palos de las cartas anteriores
\end{enumerate}
Después de que el jugador responda a cada una de las preguntas se procederá a sacar una nueva carta. En caso de fallo el jugador debe beber y volver a empezar. Se acaba una vez que el jugador halla adivinado todas o pasen 20 rondas.

\subsection{Casillas de colores}
El jugador cogerá una carta del color de la casilla y podrá elegir entre realizar la prueba descrita o beber la cantidad dispuesta. Si el jugador elige beber se procederá a girar su ficha evitando así que el jugador pueda pasar otra prueba bebiendo. En caso de que la ficha del jugador esté girada no podrá elegir beber en lugar de realizar la prueba.
\subsection{Carcel}
El jugador no puede jugar por el tiempo que esté en la carcel. En la carcel hay una estricta ley de drogas, está prohibido beber.

\section{Advertencia!!}
Se trata de un juego de beber. No os paseis, no habrá nadie para ocuparse de vosotros.

\end{document}
